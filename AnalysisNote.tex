\documentclass{cmspaper}

\usepackage{xspace}
\usepackage{xcolor}
\usepackage{cancel}
\usepackage{graphicx}
\usepackage{subcaption}
\usepackage{amssymb}
\usepackage{amsmath}
\usepackage[section]{placeins}
\usepackage{hepnames}
\usepackage{hyperref}
\usepackage{cite}
\usepackage{makecell}
\usepackage{ulem}
\usepackage{epstopdf}
\usepackage{multirow}
\usepackage{longtable}

%Add line numbers
\usepackage{lineno}
\linenumbers

\newcommand{\bea}{\begin{eqnarray}}
\newcommand{\eea}{\end{eqnarray}}

\DeclareMathOperator*{\argmin}{argmin}
\DeclareMathOperator*{\argmax}{argmax}

\renewcommand{\c}[1]{\,{\rm cos}\,#1}
\newcommand{\s}[1]{\,{\rm sin}\,#1}
\renewcommand{\cos}[1]{\,{\rm cos}(#1)}
\renewcommand{\sin}[1]{\,{\rm sin}(#1)}
\def\pslash{\not{\hbox{\kern-2pt p}}}

\definecolor{mygreen}{rgb}{0.12, 0.3, 0.17}
\definecolor{myred}{rgb}{0.45, 0.0, 0.09}
\definecolor{myblue}{rgb}{0.1, 0.1, 0.44}
\newcommand{\I}[1]{\ensuremath{\mathrm{\textcolor{mygreen}{\textbf{#1}}}}}
\newcommand{\V}[1]{\ensuremath{\mathrm{\textcolor{myblue}{\textbf{#1}}}}}
\newcommand{\D}[1]{\ensuremath{\mathrm{\textcolor{myred}{\textbf{#1}}}}}
\newcommand{\lab}{\ensuremath{\mathrm{\textbf{lab}}}}

\newcommand{\pfour}[2]{\ensuremath{\mathbf{p}_{\footnotesize #1}^{\footnotesize~#2}}}
\newcommand{\pthree}[2]{\ensuremath{\vec{p}_{\footnotesize#1}^{\footnotesize~#2}}}
\newcommand{\pone}[2]{\ensuremath{p_{\footnotesize#1}^{\footnotesize~#2}}}
\newcommand{\phat}[2]{\ensuremath{\hat{p}_{\footnotesize#1}^{\footnotesize~#2}}}
\newcommand{\E}[2]{\ensuremath{E_{\footnotesize#1}^{\footnotesize~#2}}}
\newcommand{\vbeta}[2]{\ensuremath{\vec{\beta}_{\footnotesize#1}^{\footnotesize~#2}}}
\newcommand{\sbeta}[2]{\ensuremath{\beta_{\footnotesize#1}^{\footnotesize~#2}}}
\newcommand{\hbeta}[2]{\ensuremath{\hat{\beta}_{\footnotesize#1}^{\footnotesize~#2}}}
\newcommand{\sgamma}[2]{\ensuremath{\gamma_{\footnotesize#1}^{\footnotesize~#2}}}
\newcommand{\mass}[2]{\ensuremath{m_{\footnotesize#1}^{\footnotesize~#2}}}
\newcommand{\Mass}[2]{\ensuremath{M_{\footnotesize#1}^{\footnotesize~#2}}}
\newcommand{\boost}[2]{\ensuremath{\Lambda_{\footnotesize#1}^{\footnotesize~#2}}}
\newcommand{\HT}{\ensuremath{H_{\mathrm{T}}}\xspace}
\newcommand{\met}{\ensuremath{p_{\mathrm{T}}^{\text{miss}}}\xspace}
\newcommand{\metvec}{\ensuremath{\vec{p}_{\mathrm{T}}^{\text{~miss}}}\xspace}
\newcommand{\dr}{\ensuremath{\Delta {\rm R}}\xspace}

\newcommand{\risr}{\ensuremath{R_{\rm ISR}}\xspace}
\newcommand{\ptisr}{\ensuremath{p_{\rm T}^{\rm~ISR}}\xspace}
\newcommand{\mperp}{\ensuremath{M_{\perp}}\xspace}
\newcommand{\gperp}{\ensuremath{\gamma_{\perp}}\xspace}
\newcommand{\ptcm}{\ensuremath{p_{\rm T}^{\rm~CM}}\xspace}
\newcommand{\dphicmi}{\ensuremath{\Delta \phi_{\rm CM,I}}\xspace}

\newcommand{\NjetISR}{\ensuremath{\rm N_{\rm jet}^{\rm ISR}}\xspace}
\newcommand{\Nlep}{\ensuremath{\rm N_{\rm lep}}\xspace}
\newcommand{\NjetS}{\ensuremath{\rm N_{\rm jet}^{\rm S}}\xspace}
\newcommand{\NbISR}{\ensuremath{\rm N_{\rm {\Pqb}~tag}^{\rm ISR}}\xspace}
\newcommand{\NbS}{\ensuremath{\rm N_{\rm {\Pqb}~tag}^{\rm S}}\xspace}
\newcommand{\NsvS}{\ensuremath{\rm N_{\rm SV}^{\rm S}}\xspace}
\newcommand{\etaSV}{\ensuremath{\eta_{\rm SV}^{\rm S}}\xspace}

%\epstopdfDeclareGraphicsRule{.gif}{png}{.png}{convert gif:#1 png:\OutputFile}
%\epstopdfDeclareGraphicsRule{.gif}{pdf}{.pdf}{convert gif:#1 pdf:\OutputFile}
%\AppendGraphicsExtensions{.gif}

\newcommand{\ttbar}{\ensuremath{{\PQt{}\PAQt}}\xspace}
\newcommand{\ttjets}{\ensuremath{{\PQt{}\PAQt}+\mathrm{jets}}\xspace}
\newcommand{\ttx}{\ensuremath{{\PQt{}\PAQt}\mathrm{X}+\mathrm{jets}}\xspace}
\newcommand{\zdy}{\ensuremath{\PZ/\PGg^{*}+\mathrm{jets}}\xspace}
\newcommand{\Wjets}{\ensuremath{\PW{}+\mathrm{jets}}\xspace}
\newcommand{\Znunu}{\ensuremath{\PZ(\to\nu\bar{\nu})+\mathrm{jets}}\xspace}
\newcommand{\et}{E_{\mathrm{T}}}
\newcommand{\alqed}{\alpha_{\mathrm{em}}}
\newcommand{\GF}{G_{\mathrm{F}}}
\newcommand{\mtop}{M_{\PQt{}}}
\newcommand{\mH}{M_{\PH{}}}
\newcommand{\mX}{M_{\mathrm{X}}}
\newcommand{\mW}{M_{\PW{}}}
\newcommand{\mZ}{M_{\PZ{}}}
\newcommand{\GZ}{\ensuremath{\Gamma_{\PZ}}}
\newcommand{\GW}{\ensuremath{\Gamma_{\PW}}}
\newcommand{\WW}{\ensuremath{\PWp\PWm}}
\newcommand{\ff}{\mathrm{f}\mathrm{\overline{f}}}
\newcommand{\bb}{\mathrm{b}\mathrm{\overline{b}}}
\newcommand{\Gt}{\Gamma_{\mathrm{t}}}
\newcommand{\mt}{m_{\mathrm{t}}}
\newcommand{\alr}{A_{LR}}
\newcommand{\ee}{\ensuremath{\Pep\Pem}}
\newcommand{\nZhad}{\ensuremath{n_{\PZ^{\mathrm{had}}}}}
\newcommand{\mumu}{\ensuremath{\Pgmp{}\Pgmm}}
\newcommand{\pipi}{\ensuremath{\Pgpp{}\Pgpm}}
\newcommand{\pip}{\ensuremath{\Pgpm{}\Pp}}
\newcommand{\kpi}{\ensuremath{\PKm{}\Pgpp}}
\newcommand{\sweff}{{\sin^2{\theta}^{\ell}_{\mathrm{eff}}}}
\newcommand{\sw}{\sin^2{\theta}_{\mathrm{W}}}
\newcommand{\ECM}{\sqrt{s}}
\newcommand{\percmsqs}{\mathrm{cm}^{-2}\mathrm{s}^{-1}}
\newcommand{\fbinv}{\mbox{\ensuremath{\,\text{fb}^{-1}}}\xspace}
\newcommand{\abinv}{\mbox{\ensuremath{\,\text{ab}^{-1}}}\xspace}
\newcommand{\MeV}{\ensuremath{\,\text{Me\hspace{-.08em}V}}\xspace}
\newcommand{\GeV}{\ensuremath{\,\text{Ge\hspace{-.08em}V}}\xspace}
\newcommand{\TeV}{\ensuremath{\,\text{Te\hspace{-.08em}V}}\xspace}
\newcommand{\xprime}{s^{\prime}/s}
\newcommand{\sqrtsp}{\sqrt{s}_{p}}
\newcommand{\pt}{\ensuremath{p_{\mathrm{T}}}\xspace}
\newcommand{\SingleWminus}{\ensuremath{\PWm\Pep\nu_{\Pe}}}
\newcommand{\SingleWplus}{\ensuremath{\PWp\Pem\overline{\nu}_{\Pe}}}
\newcommand{\gOneZ}{g_{1}^{\mathrm{Z}} }
\newcommand{\kgamma}{\kappa_{\gamma}}
\newcommand{\lgamma}{\lambda_{\gamma}}

% File with various sparticle combinations
%
% The PS based names are in hepnames.sty/hepnicenames.sty/heppennames.sty
%
\def \conentwo {\PScharginoOnepm \PSneutralinoTwo}
\def \conecone {\PScharginoOneplus \PScharginoOneminus}
%\def \c1c1 {\PScharginoOneplus \PScharginoOneminus}
\def \nonentwo {\PSneutralinoOne \PSneutralinoTwo}
\def \ntwontwo {\PSneutralinoTwo \PSneutralinoTwo}

\hypersetup{
    colorlinks=true,
    linkcolor=black,    %Section labels etc
    citecolor=black,    %Color of in-line citation
    filecolor=magenta,
    urlcolor=blue,      %Color of URL links for DOI and for arXiv.
}
% \urlstyle{same}

% Try to circumvent "Output loop---100 consecutive dead cycles." error
\maxdeadcycles=300
\extrafloats{100}

\begin{document}

%==============================================================================
% title page for few authors

\begin{titlepage}

% select one of the following and type in the proper number:
   \cmsnote{AN-2022-X}
%  \internalnote{2005/000}
%  \conferencereport{2005/000}
   Version 3   %Change for each major distributed version.
   \date{\today}

  \title{Material Measurements Using Photon Conversions}

  \begin{Authlist}
    Justin Anguiano,
    Graham Wilson
       \Instfoot{cern}{University of Kansas}
  \end{Authlist}

% if needed, use the following:
%\collaboration{Flying Saucers Investigation Group}
%\collaboration{CMS collaboration}

  %\Anotfoot{a}{On leave from prison}
  %\Anotfoot{b}{Now at the Moon}

\begin{abstract}

Radiation length measurements of tracker material by measuring the amount of  photons converting to electron positron pairs in the tracker volume.

\end{abstract}


% if needed, use the following:
%\conference{Presented at {\it Physics Rumours}, Coconut Island, April 1, 2005}
%\submitted{Submitted to {\it Physics Rumours}}
%\note{Preliminary version}

\end{titlepage}

\setcounter{page}{2}%JPP

%==============================================================================
% title page for many authors
%
%\begin{titlepage}
%  \internalnote{2005/000}
%  \title{CMS Technical Note Template}
%
%  \begin{Authlist}
%    A.~Author\Iref{cern}, B.~Author\Iref{cern}, C.~Author\IAref{cern}{a},
%    D.~Author\IIref{cern}{ieph}, E.~Author\IIAref{cern}{ieph}{b},
%    F.~Author\Iref{ieph}
%  \end{Authlist}
%
%  \Instfoot{cern}{CERN, Geneva, Switzerland}
%  \Instfoot{ieph}{Institute of Experimental Physics, Hepcity, Wonderland}
%  \Anotfoot{a}{On leave from prison}
%  \Anotfoot{b}{Now at the Moon}
%
%  \begin{abstract}
%    This is a template of a CMS paper, written in LaTeX,
%    processed with {\it cmspaper.sty} style.
%    It is based on the {\it cernart.sty} and {\it articlet.sty} styles.
%    There are two versions of the title page.
%    The current one is designed for many authors.
%    The one on the previous page is for few authors.
%    Just delete the one which you do not need.
%  \end{abstract}
%
%\end{titlepage}
%
%==============================================================================

\tableofcontents

%\input{tex/update-log.tex}

\clearpage

\section{Introduction \label{sec:intro}}
Introduction with this citation~\cite{Aad:2012tfa,Chatrchyan:2012ufa}


\pagebreak

\section{Data and Simulated Event Samples \label{sec:samples}}
%\input{tex/sec_samples}

\clearpage

\pagebreak

\section{Photon Conversion Reconstruction \label{sec:reco}}
%\input{tex/sec_reco}

%\subsection{Jet and Soft Vertex {\Pqb} tagging\label{sec:btag}}
%\input{tex/b_tagging}



%\input{tex/sec_RJR}

\section{Analysis Strategy and Selection \label{sec:analysis}}
%\input{tex/sec_analysis}

\section{Efficiency and Photon Flux \label{sec:effandflux}}


\section{Results \label{sec:results}}
%\input{tex/sec_results}

% Select ONE bibliography style
% tdr script includes elsarticle-num-names, lucas_unsrt, lucas_unsrt_epjc
%\bibliographystyle{unsrt}
%\bibliographystyle{elsarticle-num-names}
%\bibliographystyle{lucas_unsrt}
\bibliographystyle{lucas_unsrt_epjc}


% For easier maintenance - split bib files per section.
\bibliography{bibs/intro}
%\bibliography{bibs/intro,bibs/b_tagging,chapters_from_dissertation/allcites}

%\bibliography{intro}
%\begin{thebibliography}{9}
%  \bibitem {NOTE000} {\bf CMS Note 2005/000},
%    X.Somebody et al.,
%    {\em "CMS Note Template"}.
%\end{thebibliography}


\appendix
\section{Test Appendix \label{app:testApp}}
%\section{Lepton Objects \label{app:lepObj}}
%\input{tex/app_lepObj}

%\section{Event Cleaning Filters \label{app:filters}}
%\input{tex/app_filters}

%\section{Trigger Strategy \label{app:triggers}}
%\input{tex/app_triggers}

%\section{Secondary Vertex {\it b}-tagging data to MC comparison
%\label{app:svbapp}}
%\input{tex/app_svb}

%\input{tex/app_kin}

%\section{On the choice of the binning \label{app:binning}}
%\input{tex/app_PTISR_gamT_binning}

%\section{Category Plots \label{app:catplots}}
%\input{tex/app_catsig}

%\section{Likelihood Ratio Test \label{app:likelihood}}
%\input{tex/app_likelihood}

%\section{Control Region Fits \label{app:CRFits}}
%\input{tex/app_controlregionfits}

%\section{Additional Signal Samples \label{app:samples}}
%\input{tex/app_samples}

%\section{Systematics \label{app:systematics}}
%\input{tex/app_systematics}


\end{document}
